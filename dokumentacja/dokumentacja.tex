\documentclass[12pt,a4paper]{article}
\usepackage[OT4]{polski}
\usepackage[top=2.5cm, bottom=2.5cm, left=2cm, right=2cm]{geometry}
\usepackage{graphicx}
\usepackage{float}
\usepackage[colorlinks=true, linkcolor=blue]{hyperref}

\begin{document}

\title{Projekt na bazy danych 2023 - Dokumentacja}
\author{Paweł Packiewicz \and Adam Sochoń \and Dominik Ochej \and Bartosz Sabat \and Adrian Tomczak}
\maketitle
\tableofcontents
%%%%%%%%%%%%%%%%%%%%%%%%%%%%%%%%%%%%%%%%%%%%%%%%%%%%%
\section{Spis użytych technologii}
Podczas tworzenia projektu wykorzystane zostały różne narzędzia.
\subsection{Python}
Oraz biblioteki:
\begin{itemize}
\item \textbf{datetime},
\item \textbf{pandas} - wykorzystane podczas generowania dat,
\item \textbf{mysql} - wykorzystana do łączenia się z bazą,
\item \textbf{cvs} - wykorzystane do otwierania plików tekstowych,
\item \textbf{math},
\item \textbf{numpy},
\item \textbf{random} - wykorzystane podczas generowania różnych danych,
\item \textbf{itertools} - wykorzystane podczas analizy,
\item \textbf{matplotlib} - wykorzystane podczas generowania wykresów,
\item \textbf{os} - wykorzystane do nawigowania między plikami.
\end{itemize}

\subsection{MariaDb}
Jako dialekt MySql użyty do komunikacji z bazą danych.

\subsection{Git}
Jako narzędzie do kontroli wersji i pracy w grupie.

\section{Lista plików i opis zawartości}

\section{Instrukcja uruchamiania}

\section{Schemat projektu bazy danych}

\section{Opis zależności funkcyjnych z wyjaśnieniem}

\section{Uzasadnienie dotyczące EKNF}

\section{Trudności podczas tworzenia projektu}
\end{document}